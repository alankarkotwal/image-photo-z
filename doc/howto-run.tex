
% Default to the notebook output style

    


% Inherit from the specified cell style.




    
\documentclass{article}

    
    
    \usepackage{graphicx} % Used to insert images
    \usepackage{adjustbox} % Used to constrain images to a maximum size 
    \usepackage{color} % Allow colors to be defined
    \usepackage{enumerate} % Needed for markdown enumerations to work
    \usepackage{geometry} % Used to adjust the document margins
    \usepackage{amsmath} % Equations
    \usepackage{amssymb} % Equations
    \usepackage[mathletters]{ucs} % Extended unicode (utf-8) support
    \usepackage[utf8x]{inputenc} % Allow utf-8 characters in the tex document
    \usepackage{fancyvrb} % verbatim replacement that allows latex
    \usepackage{grffile} % extends the file name processing of package graphics 
                         % to support a larger range 
    % The hyperref package gives us a pdf with properly built
    % internal navigation ('pdf bookmarks' for the table of contents,
    % internal cross-reference links, web links for URLs, etc.)
    \usepackage{hyperref}
    \usepackage{longtable} % longtable support required by pandoc >1.10
    

    
    
    \definecolor{orange}{cmyk}{0,0.4,0.8,0.2}
    \definecolor{darkorange}{rgb}{.71,0.21,0.01}
    \definecolor{darkgreen}{rgb}{.12,.54,.11}
    \definecolor{myteal}{rgb}{.26, .44, .56}
    \definecolor{gray}{gray}{0.45}
    \definecolor{lightgray}{gray}{.95}
    \definecolor{mediumgray}{gray}{.8}
    \definecolor{inputbackground}{rgb}{.95, .95, .85}
    \definecolor{outputbackground}{rgb}{.95, .95, .95}
    \definecolor{traceback}{rgb}{1, .95, .95}
    % ansi colors
    \definecolor{red}{rgb}{.6,0,0}
    \definecolor{green}{rgb}{0,.65,0}
    \definecolor{brown}{rgb}{0.6,0.6,0}
    \definecolor{blue}{rgb}{0,.145,.698}
    \definecolor{purple}{rgb}{.698,.145,.698}
    \definecolor{cyan}{rgb}{0,.698,.698}
    \definecolor{lightgray}{gray}{0.5}
    
    % bright ansi colors
    \definecolor{darkgray}{gray}{0.25}
    \definecolor{lightred}{rgb}{1.0,0.39,0.28}
    \definecolor{lightgreen}{rgb}{0.48,0.99,0.0}
    \definecolor{lightblue}{rgb}{0.53,0.81,0.92}
    \definecolor{lightpurple}{rgb}{0.87,0.63,0.87}
    \definecolor{lightcyan}{rgb}{0.5,1.0,0.83}
    
    % commands and environments needed by pandoc snippets
    % extracted from the output of `pandoc -s`
    \DefineVerbatimEnvironment{Highlighting}{Verbatim}{commandchars=\\\{\}}
    % Add ',fontsize=\small' for more characters per line
    \newenvironment{Shaded}{}{}
    \newcommand{\KeywordTok}[1]{\textcolor[rgb]{0.00,0.44,0.13}{\textbf{{#1}}}}
    \newcommand{\DataTypeTok}[1]{\textcolor[rgb]{0.56,0.13,0.00}{{#1}}}
    \newcommand{\DecValTok}[1]{\textcolor[rgb]{0.25,0.63,0.44}{{#1}}}
    \newcommand{\BaseNTok}[1]{\textcolor[rgb]{0.25,0.63,0.44}{{#1}}}
    \newcommand{\FloatTok}[1]{\textcolor[rgb]{0.25,0.63,0.44}{{#1}}}
    \newcommand{\CharTok}[1]{\textcolor[rgb]{0.25,0.44,0.63}{{#1}}}
    \newcommand{\StringTok}[1]{\textcolor[rgb]{0.25,0.44,0.63}{{#1}}}
    \newcommand{\CommentTok}[1]{\textcolor[rgb]{0.38,0.63,0.69}{\textit{{#1}}}}
    \newcommand{\OtherTok}[1]{\textcolor[rgb]{0.00,0.44,0.13}{{#1}}}
    \newcommand{\AlertTok}[1]{\textcolor[rgb]{1.00,0.00,0.00}{\textbf{{#1}}}}
    \newcommand{\FunctionTok}[1]{\textcolor[rgb]{0.02,0.16,0.49}{{#1}}}
    \newcommand{\RegionMarkerTok}[1]{{#1}}
    \newcommand{\ErrorTok}[1]{\textcolor[rgb]{1.00,0.00,0.00}{\textbf{{#1}}}}
    \newcommand{\NormalTok}[1]{{#1}}
    
    % Define a nice break command that doesn't care if a line doesn't already
    % exist.
    \def\br{\hspace*{\fill} \\* }
    % Math Jax compatability definitions
    \def\gt{>}
    \def\lt{<}
    % Document parameters
    \title{howto-run}
    
    
    

    % Pygments definitions
    
\makeatletter
\def\PY@reset{\let\PY@it=\relax \let\PY@bf=\relax%
    \let\PY@ul=\relax \let\PY@tc=\relax%
    \let\PY@bc=\relax \let\PY@ff=\relax}
\def\PY@tok#1{\csname PY@tok@#1\endcsname}
\def\PY@toks#1+{\ifx\relax#1\empty\else%
    \PY@tok{#1}\expandafter\PY@toks\fi}
\def\PY@do#1{\PY@bc{\PY@tc{\PY@ul{%
    \PY@it{\PY@bf{\PY@ff{#1}}}}}}}
\def\PY#1#2{\PY@reset\PY@toks#1+\relax+\PY@do{#2}}

\def\PY@tok@gd{\def\PY@tc##1{\textcolor[rgb]{0.63,0.00,0.00}{##1}}}
\def\PY@tok@gu{\let\PY@bf=\textbf\def\PY@tc##1{\textcolor[rgb]{0.50,0.00,0.50}{##1}}}
\def\PY@tok@gt{\def\PY@tc##1{\textcolor[rgb]{0.00,0.25,0.82}{##1}}}
\def\PY@tok@gs{\let\PY@bf=\textbf}
\def\PY@tok@gr{\def\PY@tc##1{\textcolor[rgb]{1.00,0.00,0.00}{##1}}}
\def\PY@tok@cm{\let\PY@it=\textit\def\PY@tc##1{\textcolor[rgb]{0.25,0.50,0.50}{##1}}}
\def\PY@tok@vg{\def\PY@tc##1{\textcolor[rgb]{0.10,0.09,0.49}{##1}}}
\def\PY@tok@m{\def\PY@tc##1{\textcolor[rgb]{0.40,0.40,0.40}{##1}}}
\def\PY@tok@mh{\def\PY@tc##1{\textcolor[rgb]{0.40,0.40,0.40}{##1}}}
\def\PY@tok@go{\def\PY@tc##1{\textcolor[rgb]{0.50,0.50,0.50}{##1}}}
\def\PY@tok@ge{\let\PY@it=\textit}
\def\PY@tok@vc{\def\PY@tc##1{\textcolor[rgb]{0.10,0.09,0.49}{##1}}}
\def\PY@tok@il{\def\PY@tc##1{\textcolor[rgb]{0.40,0.40,0.40}{##1}}}
\def\PY@tok@cs{\let\PY@it=\textit\def\PY@tc##1{\textcolor[rgb]{0.25,0.50,0.50}{##1}}}
\def\PY@tok@cp{\def\PY@tc##1{\textcolor[rgb]{0.74,0.48,0.00}{##1}}}
\def\PY@tok@gi{\def\PY@tc##1{\textcolor[rgb]{0.00,0.63,0.00}{##1}}}
\def\PY@tok@gh{\let\PY@bf=\textbf\def\PY@tc##1{\textcolor[rgb]{0.00,0.00,0.50}{##1}}}
\def\PY@tok@ni{\let\PY@bf=\textbf\def\PY@tc##1{\textcolor[rgb]{0.60,0.60,0.60}{##1}}}
\def\PY@tok@nl{\def\PY@tc##1{\textcolor[rgb]{0.63,0.63,0.00}{##1}}}
\def\PY@tok@nn{\let\PY@bf=\textbf\def\PY@tc##1{\textcolor[rgb]{0.00,0.00,1.00}{##1}}}
\def\PY@tok@no{\def\PY@tc##1{\textcolor[rgb]{0.53,0.00,0.00}{##1}}}
\def\PY@tok@na{\def\PY@tc##1{\textcolor[rgb]{0.49,0.56,0.16}{##1}}}
\def\PY@tok@nb{\def\PY@tc##1{\textcolor[rgb]{0.00,0.50,0.00}{##1}}}
\def\PY@tok@nc{\let\PY@bf=\textbf\def\PY@tc##1{\textcolor[rgb]{0.00,0.00,1.00}{##1}}}
\def\PY@tok@nd{\def\PY@tc##1{\textcolor[rgb]{0.67,0.13,1.00}{##1}}}
\def\PY@tok@ne{\let\PY@bf=\textbf\def\PY@tc##1{\textcolor[rgb]{0.82,0.25,0.23}{##1}}}
\def\PY@tok@nf{\def\PY@tc##1{\textcolor[rgb]{0.00,0.00,1.00}{##1}}}
\def\PY@tok@si{\let\PY@bf=\textbf\def\PY@tc##1{\textcolor[rgb]{0.73,0.40,0.53}{##1}}}
\def\PY@tok@s2{\def\PY@tc##1{\textcolor[rgb]{0.73,0.13,0.13}{##1}}}
\def\PY@tok@vi{\def\PY@tc##1{\textcolor[rgb]{0.10,0.09,0.49}{##1}}}
\def\PY@tok@nt{\let\PY@bf=\textbf\def\PY@tc##1{\textcolor[rgb]{0.00,0.50,0.00}{##1}}}
\def\PY@tok@nv{\def\PY@tc##1{\textcolor[rgb]{0.10,0.09,0.49}{##1}}}
\def\PY@tok@s1{\def\PY@tc##1{\textcolor[rgb]{0.73,0.13,0.13}{##1}}}
\def\PY@tok@sh{\def\PY@tc##1{\textcolor[rgb]{0.73,0.13,0.13}{##1}}}
\def\PY@tok@sc{\def\PY@tc##1{\textcolor[rgb]{0.73,0.13,0.13}{##1}}}
\def\PY@tok@sx{\def\PY@tc##1{\textcolor[rgb]{0.00,0.50,0.00}{##1}}}
\def\PY@tok@bp{\def\PY@tc##1{\textcolor[rgb]{0.00,0.50,0.00}{##1}}}
\def\PY@tok@c1{\let\PY@it=\textit\def\PY@tc##1{\textcolor[rgb]{0.25,0.50,0.50}{##1}}}
\def\PY@tok@kc{\let\PY@bf=\textbf\def\PY@tc##1{\textcolor[rgb]{0.00,0.50,0.00}{##1}}}
\def\PY@tok@c{\let\PY@it=\textit\def\PY@tc##1{\textcolor[rgb]{0.25,0.50,0.50}{##1}}}
\def\PY@tok@mf{\def\PY@tc##1{\textcolor[rgb]{0.40,0.40,0.40}{##1}}}
\def\PY@tok@err{\def\PY@bc##1{\fcolorbox[rgb]{1.00,0.00,0.00}{1,1,1}{##1}}}
\def\PY@tok@kd{\let\PY@bf=\textbf\def\PY@tc##1{\textcolor[rgb]{0.00,0.50,0.00}{##1}}}
\def\PY@tok@ss{\def\PY@tc##1{\textcolor[rgb]{0.10,0.09,0.49}{##1}}}
\def\PY@tok@sr{\def\PY@tc##1{\textcolor[rgb]{0.73,0.40,0.53}{##1}}}
\def\PY@tok@mo{\def\PY@tc##1{\textcolor[rgb]{0.40,0.40,0.40}{##1}}}
\def\PY@tok@kn{\let\PY@bf=\textbf\def\PY@tc##1{\textcolor[rgb]{0.00,0.50,0.00}{##1}}}
\def\PY@tok@mi{\def\PY@tc##1{\textcolor[rgb]{0.40,0.40,0.40}{##1}}}
\def\PY@tok@gp{\let\PY@bf=\textbf\def\PY@tc##1{\textcolor[rgb]{0.00,0.00,0.50}{##1}}}
\def\PY@tok@o{\def\PY@tc##1{\textcolor[rgb]{0.40,0.40,0.40}{##1}}}
\def\PY@tok@kr{\let\PY@bf=\textbf\def\PY@tc##1{\textcolor[rgb]{0.00,0.50,0.00}{##1}}}
\def\PY@tok@s{\def\PY@tc##1{\textcolor[rgb]{0.73,0.13,0.13}{##1}}}
\def\PY@tok@kp{\def\PY@tc##1{\textcolor[rgb]{0.00,0.50,0.00}{##1}}}
\def\PY@tok@w{\def\PY@tc##1{\textcolor[rgb]{0.73,0.73,0.73}{##1}}}
\def\PY@tok@kt{\def\PY@tc##1{\textcolor[rgb]{0.69,0.00,0.25}{##1}}}
\def\PY@tok@ow{\let\PY@bf=\textbf\def\PY@tc##1{\textcolor[rgb]{0.67,0.13,1.00}{##1}}}
\def\PY@tok@sb{\def\PY@tc##1{\textcolor[rgb]{0.73,0.13,0.13}{##1}}}
\def\PY@tok@k{\let\PY@bf=\textbf\def\PY@tc##1{\textcolor[rgb]{0.00,0.50,0.00}{##1}}}
\def\PY@tok@se{\let\PY@bf=\textbf\def\PY@tc##1{\textcolor[rgb]{0.73,0.40,0.13}{##1}}}
\def\PY@tok@sd{\let\PY@it=\textit\def\PY@tc##1{\textcolor[rgb]{0.73,0.13,0.13}{##1}}}

\def\PYZbs{\char`\\}
\def\PYZus{\char`\_}
\def\PYZob{\char`\{}
\def\PYZcb{\char`\}}
\def\PYZca{\char`\^}
\def\PYZsh{\char`\#}
\def\PYZpc{\char`\%}
\def\PYZdl{\char`\$}
\def\PYZti{\char`\~}
% for compatibility with earlier versions
\def\PYZat{@}
\def\PYZlb{[}
\def\PYZrb{]}
\makeatother


    % Exact colors from NB
    \definecolor{incolor}{rgb}{0.0, 0.0, 0.5}
    \definecolor{outcolor}{rgb}{0.545, 0.0, 0.0}



    
    % Prevent overflowing lines due to hard-to-break entities
    \sloppy 
    % Setup hyperref package
    \hypersetup{
      breaklinks=true,  % so long urls are correctly broken across lines
      colorlinks=true,
      urlcolor=blue,
      linkcolor=darkorange,
      citecolor=darkgreen,
      }
    % Slightly bigger margins than the latex defaults
    
    \geometry{verbose,tmargin=1in,bmargin=1in,lmargin=1in,rmargin=1in}
    
    

    \begin{document}
    
    
    \maketitle
    
    

    
    \section{A how-to to running image-photo-z}

\begin{center}\rule{3in}{0.4pt}\end{center}

\emph{A guide to installation and running the package with the main
file}

\subsection{Installation:}

Installing the dependencies and cloning the repository is enough to set
up the environment to run the package. For installing the dependencies,
the setup-deps script provided in the root folder of the package should
work on a system that can use apt-get. You need administrator
priviledges to run this script. Run it as follows on the system shell.

    \begin{Verbatim}[commandchars=\\\{\}]
{\color{incolor}In [{\color{incolor}6}]:} \PY{n}{cd} \PY{n}{image}\PY{o}{-}\PY{n}{photo}\PY{o}{-}\PY{n}{z}\PY{o}{/}
\end{Verbatim}

    \begin{Verbatim}[commandchars=\\\{\}]
/home/aloo/image-photo-z
    \end{Verbatim}

    \begin{Verbatim}[commandchars=\\\{\}]
{\color{incolor}In [{\color{incolor}}]:} \PY{o}{.}\PY{o}{/}\PY{n}{setup}\PY{o}{-}\PY{n}{deps}\PY{o}{.}\PY{n}{sh}
\end{Verbatim}

    If for some reason you are not able to or do not wish to run the setup
script, here is a list of dependencies (in order of their requirement
for the pipeline) that need to be installed. Unless mentioned, it is
best to install dependencies in their default locations using the
standard installer provided or the standard package management for your
system
Python \textgreater{}= v2.7. python-dev, the development package, needs
to be installed as well.
Numpy \textgreater{}= v1.6.1
Scipy \textgreater{}= v0.14.0
Astropy \textgreater{}= v0.3.2
Montage Image Mosaic Software \textgreater{}= v3.3. Best installed by
downloading and compiling the source package with GNU Make.
montage\_wrapper \textgreater{}= v0.9.7
SExtractor \textgreater{}= v2.19.5. In my experience, gives trouble if
installed via any source other than the rpm package on the Astromatic
website.
Scikit-learn \textgreater{}= v0.15.0b1
MLZ \textgreater{}= v0.0.1, best installed via pip.
The procedure for compiling source packages using make is the following,
on the system shell:

    \begin{Verbatim}[commandchars=\\\{\}]
{\color{incolor}In [{\color{incolor}}]:} \PY{n}{cd} \PY{p}{[}\PY{n}{package}\PY{o}{-}\PY{n}{name}\PY{p}{]}        \PY{c}{\PYZsh{} This is the root directory for the source package. }
\end{Verbatim}

    \begin{Verbatim}[commandchars=\\\{\}]
{\color{incolor}In [{\color{incolor}}]:} \PY{n}{cmake} \PY{o}{.} \PY{p}{[}\PY{n}{OR}\PY{p}{]} \PY{o}{.}\PY{o}{/}\PY{n}{configure} \PY{c}{\PYZsh{} This depends on the package. If it contains a configure file, use ./configure.}
\end{Verbatim}

    \begin{Verbatim}[commandchars=\\\{\}]
{\color{incolor}In [{\color{incolor}}]:} \PY{n}{make}                     \PY{c}{\PYZsh{} This compiles the package.}
\end{Verbatim}

    \begin{Verbatim}[commandchars=\\\{\}]
{\color{incolor}In [{\color{incolor}}]:} \PY{n}{sudo} \PY{n}{make} \PY{n}{install}        \PY{c}{\PYZsh{} Optional, need admin priviledges for this. Places binaries in their default positions.}
\end{Verbatim}

    For installing via rpm you need the rpm package management software.
Once you have that, you do

    \begin{Verbatim}[commandchars=\\\{\}]
{\color{incolor}In [{\color{incolor}}]:} \PY{n}{sudo} \PY{n}{rpm} \PY{o}{-}\PY{n}{i} \PY{p}{[}\PY{n}{package}\PY{o}{-}\PY{n}{name}\PY{p}{]}\PY{o}{.}\PY{n}{rpm}
\end{Verbatim}

    Installing via pip is accomplished as follows:

    \begin{Verbatim}[commandchars=\\\{\}]
{\color{incolor}In [{\color{incolor}}]:} \PY{n}{sudo} \PY{n}{pip} \PY{n}{install} \PY{p}{[}\PY{n}{package}\PY{o}{-}\PY{n}{name}\PY{p}{]}
\end{Verbatim}

    \subsection{Creating the config file}

The config file is the set of input parameters to the code. Most
parameter names are self-explanatory. Examples are provided in the
config.cfg.template file provided in the root folder. They are explained
here for completeness:\\For yes/no parameters, specify `yes' or `no'
without the quotes. This is \textbf{case-sensitive}. For an explanation
of what the files described below are, refer to the ``Getting to know
the pipeline'' document.

\begin{itemize}
\item
  IMAGE\_PHOTOZ\_PATH: This is the path to your local image-photo-z
  installation. My value for this is /home/aloo/image-photo-z.
\item
  USE\_MPI: Set this to yes if you want to use MPI for the computation.
\item
  N\_PROCESSORS: Set this to the number of processors you want to use,
  if using MPI.
\item
  TRAINING\_CATALOG: The object catalog file for training objects. This
  file must be in a specified format as shown in the example catalog
  file one\_square\_degree.csv provided with the package. For more
  information on generating this file see the pipeline description.
\item
  TESTING\_CATALOG: The object catalog file for testing objects. This is
  similar to the earlier file.
\item
  TRAINING\_CATALOG\_PROCESSED: The name of the processed catalog file
  generated by the program.
\item
  TESTING\_CATALOG\_PROCESSED: The name of the processed catalog file
  generated by the program.
\item
  BANDS: Filters to be used. Specify these as comma-separated without
  spaces.
\item
  REMAKE\_CATALOGS: Specify whether or not to regenerate the catalogs by
  querying the SDSS SAS.
\item
  REGENERATE\_PIXEL\_DATA: Specify whether or not to (re)generate pixel
  data vectors.
\item
  LOG\_INDEPENDENTLY: Specify whether or not to generate a list
  (logfile) of run-camcol-field combinations independent of image
  downloading.
\item
  LOGFILE: Name of the logfile used to record run-camcol-field
  combinations in the catalog.
\item
  LOCAL\_IMAGES: Specify if the images are stored locally or not.
  Alternatively, select whether or not to download images (again).
\item
  TRAINING\_IMAGES\_DIR: If images are local, this is the directory in
  which training images are stored. The images need to be stored as
  {[}run{]}-{[}camcol{]}-{[}field{]}-{[}band{]}.fits (3813-5-24-u.fits,
  for example) in this directory.
\item
  TESTING\_IMAGES\_DIR: Similar to the above for testing images.
\item
  PROCESSING\_DIR: The directory in which all processing is to be done.
  Images per frame are moved into this directory, all data is extracted
  from the images. Temporary files generated by the process are also
  stored here.
\item
  TRAINING\_CLASSIFIED\_DATA\_DIR: The code generates classified
  training data into this directory. Depending on which source types
  (galaxies, stars, ..) are used, the data is stored in subfolders
  inside this folder.
\item
  TESTING\_CLASSIFIED\_DATA\_DIR: Similar to the above, with testing
  data.
\item
  TRAINING\_DATA\_FILE: Data collected from all sources is collected
  into this file for training.
\item
  TESTING\_DATA\_FILE: Similar to the above for testing.
\item
  TRAINING\_TARGET\_FILE: If using kNN, the redshifts are stored in this
  file before training.
\item
  TESTING\_TARGET\_FILE: Similar to the above, for testing.
\item
  TESTING\_PREDICTION\_FILE: If using kNN, immediately post testing, the
  output for respective predictions are stored in this file.
\item
  TRAIN\_AND\_TEST: Enable or disable training and testing.
\item
  PROBLEM\_TYPE: Select whether you want to do classification or
  regression.
\item
  USE\_GALAXIES: Enable or disable using galaxy pixels.
\item
  USE\_STARS: Enable or disable using star pixels.
\item
  USE\_QSOS: Enable or disable using QSO pixels.
\item
  USE\_BACKGROUND: Enable or disable using background pixels.
\item
  NUMBER\_NEIGHBORS: If using kNN, this sets the number of neighbors to
  be used.
\item
  KNN\_OUTPUT\_FILE: If using kNN, this sets the final output filename
  from which final plots may be generated.
\item
  KNN\_ROUND\_OFF\_CLASSIFICATION: If using kNN, this sets whether or
  not to round off the classification result.
\item
  CLEAN\_AFTER\_DONE: Set to yes if only the output file and the
  downloaded images are to be kept after the pipeline finishes.
\item
  CLEAN\_ON\_INTERRUPT: Set to yes if generated temporary files and data
  is to be erased when the process is interrupted.
\item
  REMOVE\_IMAGES\_AFTER\_DONE: Set this to yes if images are to be
  deleted after the program finishes.
\item
  REMOVE\_IMAGES\_ON\_INTERRUPT: Set this to yes if images are to be
  deleted when the process is interrupted.
\item
  REMOVE\_INTERMEDIATE\_IMAGES: Set this to yes if intermediate images
  (like registered and error images) are to be deleted or no if they are
  to be moved to another location specified by
  INTERMEDIATE\emph{*}FILES.
\item
  INTERMEDIATE\_TRAINING\_FILES: Directory to store intermediate files
  if REMOVE\_INTERMEDIATE\_IMAGES is set to no.
\item
  TIME: Set whether or not to time the process. \emph{Not implemented
  yet}.
\item
  TIME\_DETAIL: Select the level of timing detail required.
\item
  TIME\_FILE: File to dump timing information in.
\end{itemize}
Refer to MLZ documentation for the meaning of MLZ config options. These
go straight to the MLZ inputfile.

Note that in case of directories, do \textbf{NOT} put a `/' at the end
of the parameter. The code does this automatically. Also, do
\textbf{NOT} delete any entry from the config file, even if it is
irrelevant to the configuration. This is done to prevent possible
KeyErrors. Do \textbf{NOT} put spaces in any of the parameters.

    \subsection{Running the pipeline:}

Using the main file with an appropriate config file is the most
straightforward way to run the pipeline end-to-end. Here, you have two
options, depending on whether you want to or not to use MPI.

This assumes you have the training and testing catalogs in the required
format. Refer to the pipeline description for how to generate these.

\begin{enumerate}[1)]
\item
  Not using MPI: If you cannot or do not wish to use MPI, the executable
  main.py is to be run on the system shell, as follows:
\end{enumerate}

    \begin{Verbatim}[commandchars=\\\{\}]
{\color{incolor}In [{\color{incolor}4}]:} \PY{n}{cd} \PY{n}{image}\PY{o}{-}\PY{n}{photo}\PY{o}{-}\PY{n}{z}
\end{Verbatim}

    \begin{Verbatim}[commandchars=\\\{\}]
/home/aloo/image-photo-z
    \end{Verbatim}

    Note that the path after the cd should be the path to \emph{your local
image-photo-z installation}. Mine is installed at
/home/aloo/image-photo-z, so its where it is. Once this is done, just
run the main on the shell as follows:

    \begin{Verbatim}[commandchars=\\\{\}]
{\color{incolor}In [{\color{incolor}}]:} \PY{n}{python} \PY{n}{main}\PY{o}{.}\PY{n}{py}
\end{Verbatim}

    This runs the code with ./config.cfg as the config file. To specify a
different config file run

    \begin{Verbatim}[commandchars=\\\{\}]
{\color{incolor}In [{\color{incolor}}]:} \PY{n}{python} \PY{n}{main}\PY{o}{.}\PY{n}{py} \PY{p}{[}\PY{n}{config}\PY{o}{-}\PY{n+nb}{file}\PY{o}{-}\PY{n}{name}\PY{p}{]}
\end{Verbatim}

    This \emph{should} get the code running. If you do get any errors at
this stage, \textbf{do report them}.

\begin{enumerate}[1)]
\setcounter{enumi}{1}
\item
  Using MPI: The file main-mpi.py needs to be run with mpirun in this
  case. Make sure that the N\_PROCESSORS variable is set to the number
  of processors you are willing to allocate to the computation in the
  config file.
\end{enumerate}

    \begin{Verbatim}[commandchars=\\\{\}]
{\color{incolor}In [{\color{incolor}}]:} \PY{n}{mpirun} \PY{o}{-}\PY{n}{np} \PY{p}{[}\PY{n}{N\PYZus{}PROCESSORS}\PY{p}{]} \PY{n}{python} \PY{n}{main}\PY{o}{-}\PY{n}{mpi}\PY{o}{.}\PY{n}{py}
\end{Verbatim}

    To specify a custom config file:

    \begin{Verbatim}[commandchars=\\\{\}]
{\color{incolor}In [{\color{incolor}}]:} \PY{n}{mpirun} \PY{o}{-}\PY{n}{np} \PY{p}{[}\PY{n}{N\PYZus{}PROCESSORS}\PY{p}{]} \PY{n}{python} \PY{n}{main}\PY{o}{-}\PY{n}{mpi}\PY{o}{.}\PY{n}{py} \PY{p}{[}\PY{n}{config}\PY{o}{-}\PY{n+nb}{file}\PY{o}{-}\PY{n}{name}\PY{p}{]}
\end{Verbatim}

    The run script in the root folder is already set to run this. To avoid
the hassle of typing all this, you can run this instead:

    \begin{Verbatim}[commandchars=\\\{\}]
{\color{incolor}In [{\color{incolor}}]:} \PY{o}{.}\PY{o}{/}\PY{n}{run}
\end{Verbatim}

    \emph{having made sure} that N\_PROCESSORS is set. This takes config.cfg
as the config file.

    The code \emph{does not} print anything out on the terminal except for
the generating training data stage.


    % Add a bibliography block to the postdoc
    
    
    
    \end{document}
